\documentclass[14pt]{extarticle}
\usepackage[utf8]{inputenc}
\usepackage{amsmath}
\usepackage[left=0.5in, right=0.5in, top=1in, bottom=1in]{geometry}
\usepackage[colorlinks=true, linkcolor=blue, urlcolor=red]{hyperref}
\usepackage{times} % Times New Roman

\begin{document}
\textbf{The total time therefore is O(x + 2x + . . . + nx). This reduces to O(x$n^2$)}\\
\textbf{Why is it O($xn^2$)? Because 1 + 2 + ... + n equals n(n+1) / 2, or O($n^2$).}\\

A common approach in string manipulation problems is to edit the string starting from 
the end and working backwards. This is useful because we have an extra buffer at the end,
which allows us to change characters without worrying about what we're overwriting.

\end{document}
